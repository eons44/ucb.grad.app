\documentclass[12pt]{article}
\usepackage[margin=1in]{geometry}
\usepackage{fancyhdr}

\setlength{\parskip}{0.75em}

\pagestyle{fancy}
\fancyhf{} % Clear default header and footer
\fancyfoot[L]{\textit{Séon O'Shannon} - \textit{Personal Statement}}
\fancyfoot[R]{\thepage}
\renewcommand{\headrulewidth}{0pt} % Remove the top line

\begin{document}

\begin{center}
    \Large{\textbf{Personal Statement for Séon O'Shannon}} \\
    University of Colorado Boulder - Biochem Ph.D.
\end{center}

\vspace{0.5cm}

Reflected in my life's journey is a commitment to multidisciplinary advancement toward my goal of extending the healthspan of peoples across the globe. A Ph.D. in biochemistry will allow me to garner the capacity to do this. As I have vascillated between computer and life sciences, I have found an untapped wealth of innovation in the tech startup world that has not made its way to research - academic or private. With a Ph.D., laboratory prowess, and the right support, I will endeavor to assist in building a lab that embraces agile methodologies and automation such that future researchers can enmesh themselves in a collaborative workplace with machines.

Accomplishing this ambitious goal will require technological advancement through a people-first, community-centered mindset - a hallmark of CU Boulder's academic community. The IQ Biology program seems apropos of someone with a diverse background to immerse themselves in the university and find the place they can do the most good. My aspirations resonate with the interdisciplinary works of Dr. Halil Aydin, Dr. Karolin Luger, and Dr. Alexandra Whiteley in particular.

I would be eager to explore the intricacies of neuronal morphology under Dr. Aydin. Working with him would allow me to translate the expertise I gained under Dr. Gunnar Newquist, of designing neurons in silico, to practical applications addressing neurodegenerative diseases. I partnered with Dr. Gunnar Newquist in the second year of my bachelor's degree at the University of Nevada, Reno (UNR). Together, we founded a startup building toys for kids with software that mimicked the neuronal behavior of Drosophila melanogaster, C. elegans, and other well studied systems. Dr. Aydin's work on neuronal morphology would dovetail well with this work.

My thesis mentor, Dr. Amy Curran, was the COO for the biotech company ZeaKal. She trained me in vector and primer design to effectively run electrophoresis gels. I was taught cloning and transformation to move our genetic constructs into E. Coli, agrobacterium, and then into several plant species, namely \textit{Arabidopsis thaliana} and \textit{Glycine max}. Much of the plant care we did was simply to identify homozygous lines. So, I quite enjoyed it when we got to evaluate the expressed performance of our constructs through fatty acid methyl ester gas chromatography (FAME GC). We qualitatively normalized the GC results to gene expression as measured by western blot. We also got to do some exploratory assays with nile red staining and fluorescence spectroscopy.

Under Dr. Curran, I learned a lot about genetic engineering and phenotypic expression; however, I have not gotten to explore the functional effects of peptide conformation. Dr. Luger's cancer-focused research would contextualize my studies with Dr. Curran into the domain of human healthcare. By explicating the organizational and regulatory mechanisms of chromatin through crystallography, I would gain hands-on experience over genetic engineering from DNA to phenotype. Replete with these skills, I would be very well positioned for a successful career fighting cancer and human senescence.

Grit in the face of difficulty was quintessential to achieving my undergraduate degree with a thesis while pursuing a startup. Putting myself through school without financial support and finding my way to a profitable career in science required that I be more than simply studious. For further business acumen, I drafted the bylaws, joined the board, and formed a 501(c)(3) from the Reno Video Game Symphony. To make ends meet, I took a job doing website development and did extra work sweeping greenhouses and washing dishes for my lab. In the end, I found myself with 4 jobs while attending school full-time. After only 3 and a half years, I graduated with a thesised B.S. degree in Biochemistry and Molecular Biology and a minor in Mathematics.

After college, I returned to work for Dr. Curran and ZeaKal as a Research Associate. At UNR, I was enamored with assisting the graduate students with their genomics pipelines, open-cv based image quantification, and other intriguing projects. Through my employment, I became respected as an expert on western blotting and accepted a role mentoring undergraduates. I was sought as an instructor, both academically and professionally, for my patience and ability to map ideas to relatable concepts. For example, while attending one undergraduate seminar on bioinformatics, the instructor, Dr. Richard Tillet, made me an impromptu TA. Despite having no knowledge of the lesson plan, I was able to provide guidance to struggling students and write out the commands that would be needed for each upcoming topic.

Beyond assisting students, I also leveraged my technical expertise for ZeaKal. I collaborated with a colleague at Amazon to develop a globally distributed cloud architecture that eased our team's remote data sharing and collaboration. Additionally, I repaired some broken growth chambers and designed dedicated hardware for monitoring the environment of our tobacco. These multidisciplinary efforts at ZeaKal have prepared me to meaningfully contribute to Dr. Whiteley's investigations into Ubiquilins. Her lab's exploration of disparate cellular systems would embolden me to discover novel approaches to human disease and senescence.

Following the pandemic, I left ZeaKal to build a business of my own. Since then, I have advised startups and worked as a senior software engineer. I was responsible for managing production web services, implementing data analytics systems, mentoring other engineers, and collaborating across departments. Today, I lead a cloud technologies startup which supports my long-term efforts of building an open-source framework for biomimicry and an accessible programming language.

My professional journey through private sector biotech and software has uniquely equipped me to succeed in my next step into academia. CU Boulder offers unparalleled opportunities for unifying my proficiencies in science and technology to advance my career, empowering me to help our species assume longevity. I am eager to forge lasting relationships with the campus faculty and community while effecting interdisciplinary advancements. I am confident that I will thrive when my commitment to innovative, compassionate research is immersed in the collaborative spirit of the University of Colorado Boulder.

\end{document}
\documentclass[12pt]{article}
\usepackage[margin=1in]{geometry}
\usepackage{fancyhdr}

\setlength{\parskip}{1em} 

\pagestyle{fancy}
\fancyhf{} % Clear default header and footer
\fancyfoot[L]{\textit{Séon O'Shannon} - \textit{IQ Biology Statement of Interest}}
\fancyfoot[R]{\thepage}
\renewcommand{\headrulewidth}{0pt} % Remove the top line

\begin{document}

\begin{center}
    \Large{\textbf{IQ Biology Statement of Interest for Séon O'Shannon}} \\
    University of Colorado Boulder - Biochem Ph.D.
\end{center}

\vspace{0.5cm}

I firmly believe the IQ Biology program would be a perfect fit for me. Interdisciplinary collaboration is part of everything I do and something I find deeply fulfilling.

While studying metabolic regulation in my undergraduate program, I was curious about the possibility of replinishing NADH extracellularly. Lipids have a very high energy density and glycogen is a fantastic capacitor. So, if we could supply NAD+ with electrons from an external source and shuttle them into a cell or protein complex that would run through basic gluconeogenisis, and if we could run that process backward, we would have the bones of a rechargable bio battery. Apparently, humans have already accomplished something akin to this with sugar batteries, but these sugar batteries lack the ability to both recharge and condense the simple sugars into more energy dense oils.

For nearly a decade I have been chipping away at an endeavor I call Develop Biology. The goal is to get software engineers to think like molecular biologists and vice vera. It started as a framework for biomimicry of molecular neuroscience and was used by my first startup: Brain2Bot. Since then, it has become a place for me to organize my thoughts on how cells do, might, and should work. Through heavily templated, backwards-compatible C++, the framework as a whole takes on an almost philosophical opinion by melding the ideas of "to have", "to do", and "to be". For example, an Atom which Covalently Bonds to a Vesicle can be treated as though it were that Vesicle. The more nuanced implications of this system are expounded by my python framework: eons (also the name of my company). When using the eons system, everything is a Functor, which is to say both a function and a class, i.e. "does" and "is". This philosophy has rapidly accelerated prototyping on my cloud platform. For Zywave, a (late) friend and I took a weekend to build an app that should theoretically have saved over \$800,000 / year by leveraging my functor-based tools.

I am now working on a language I call Elder or elderlang (i.e. the Eons Language of Development for Entropic Reduction). The goal is to integrate the Develop Biology C++ code into the eons python framework, then use elderlang to wrap the python code in an easy-to-use and quick-to-learn language. Accomplishing this would make the highly technical concepts I elucidate in Develop Biology and make them accessible to a wide audience, including (hopefully) non-English speakers.

Everything I build and commission is open source and freely available. So, if accepted to the IQ Biology program and there's interest in using my tools and not just my expertise, I would be more than happy to provide training and advice.

I will always seek to cultivate diversity and interdisciplinary thinking. However, the IQ Biology program seems like the incubator where ideas blending computer science and biology flourish. I would absolutely adore working across fields with talented teams to build collaborative, impactful solutions.

\end{document}